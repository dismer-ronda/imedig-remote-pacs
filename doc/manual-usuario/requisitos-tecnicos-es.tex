\documentclass{plantilla-manual-usuario}


\autor{PRYADES Soluciones Informáticas SL}
\proyecto{Teleradiología en la Nube}
\logotipo{images/imedig-logo.jpg}
\title{Manual de Usuario}
\company{images/pryades-logo.png}
\date{02/09/2015}

\begin{document}

\frontmatter

\maketitle

\newpage

\mainmatter

\section{Requerimientos de acceso a Internet para la pasarela 3G}

Para que la pasarela 3G IMEDIG mantenga una conectividad a Internet adecuada y pueda ofrecer sus servicios eficientemente se requiere de una tarjeta SIM habilitada para un operador local, con un contrato de datos en el que la velocidad de transmisión/recepción permanezca constante independientemente del volumen de datos. Los planes de datos en algunos países incluyen una cantidad de datos a máxima velocidad, por ejemplo 1 GB, y a partir de ese volumen la velocidad disminuye a valores que no son adecuados para la transmisión de imágenes. Se deberá tener cuidado en la selección del plan de datos para evitar esa situación.  

\section{Requerimientos de equipamiento y acceso a Internet para las estaciones de visualización de imágenes}

Para el acceso remoto a las imágenes almacenadas en la pasarela 3G se requiere de un equipo informático con las siguientes características mínimas:

\begin{itemize}
\item Velocidad de CPU 2 GHz
\item Monitor con resolución mínima de 1280 x 1024 pixels, apto para diagnóstico médico
\item Memoria RAM 2 GB
\item Navegador Internet Firefox, Chrome, Safari, Internet Explorer 9 u Opera
\item Conexión a Internet de 1 Mbps
\item Sistema operativo Windows XP o superior, Linux o Macintosh
\end{itemize}

\section{Contacto}

Para más informacion contactar con:

PRYADES Soluciones Informáticas SL\\*
c/ La Fragua 5\\*
28260, Galapagar, Madrid\\*
Spain\\*
Ph: +34 918 586 353\\*
Correo: direccion@pryades.com\\*

\end{document}
